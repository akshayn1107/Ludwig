\documentclass[11pt]{article}

\usepackage{fullpage}
\usepackage{proof}
\setlength{\inferLineSkip}{4pt}
\usepackage{latexsym}
\usepackage[breaklinks=true,
  colorlinks=true,
  citecolor=blue,
  linkcolor=blue,
  urlcolor=blue]{hyperref}

\begin{document}
\title{15-411 Compiler Design, Fall 2013\\ Lab 4}
\author{Instructor: Frank Pfenning\\TAs: Robbie Harwood, Sri Raghavan,
  Max Serrano}
\date{Test Programs Due: 11:59pm, Thursday, October 24, 2013\\
Compilers Due: 11:59pm, Thursday, October 31, 2013}
\maketitle

% \newcommand{\danger}{\marginpar[\hfill\dbend]{\dbend\hfill}}
\newcommand{\danger}{\textbf{!!!}}

\newcommand{\nonterm}[1]{$\langle${#1}$\rangle$}
\newcommand{\tok}[1]{$\langle$\emph{#1}$\rangle$}
\newcommand{\term}[1]{\textbf {#1}}
\newcommand{\OR}{\ensuremath{\ | \ \ }}

%% various commands taken from the 15-312 assignment handouts
\newcommand{\proves}{\vdash}

\newcommand{\G}{\Gamma}
\newcommand{\cons}[2]{#1, \, #2}
\newcommand{\typed}[2]{#1 : #2}
\newcommand{\valid}[1]{#1 \; \mathit{valid}}
\newcommand{\typof}[3]{{#1} \proves \typed{#2}{#3}}

%% L3
\newcommand{\tint}{\mathbf{int}}
\newcommand{\tbool}{\mathbf{bool}}
\newcommand{\etrue}{\mathbf{true}}
\newcommand{\efalse}{\mathbf{false}}
\newcommand{\eintconst}{\mathbf{intconst}}
\newcommand{\sassn}[2]{\mathbf{assign}(#1, #2)}
\newcommand{\sif}[3]{\mathbf{if}(#1, #2, #3)}
\newcommand{\swhile}[2]{\mathbf{while}(#1, #2)}
\newcommand{\sfor}[4]{\mathbf{for}(#1, #2, #3, #4)}
\newcommand{\scont}{\mathbf{continue}}
\newcommand{\sbreak}{\mathbf{break}}
\newcommand{\sret}[1]{\mathbf{return}(#1)}
\newcommand{\snop}{\mathbf{nop}}
\newcommand{\sseq}[2]{\mathbf{seq}(#1, #2)}
\newcommand{\sdecl}[3]{\mathbf{declare}(#1, #2, #3)}

\newcommand{\snil}{\mathbf{nil}}
\newcommand{\sextfdecl}[2]{\mathbf{extfdecl}(#1, #2)}
\newcommand{\sintfdecl}[2]{\mathbf{intfdecl}(#1, #2)}
\newcommand{\sfun}[3]{\mathbf{fun}(#1, #2, #3)}

%%%%%%%%%%%%%%%%%%%%%%%%%%%%%%%%%%%%%%%%%%%%%%%%%%%%%%%%%%%%%%%%%%%%%%%%%%%%%%

\section{Introduction}

The goal of the lab is to implement a complete compiler for the
language L4.  This language extends L3 with pointers, arrays, and
structs.  With the ability to store global state, you should be able
to write a wide variety of interesting programs.  As always,
correctness is paramount, but you should take care to make sure your
compiler runs in reasonable time.

% A special ``dangerous bend'' symbol\footnote{See Knuth, The {\TeX}book, 9th
% printing, 1986} marks some particularly important
% warnings.  These warnings are about ``epic fail'' issues, not ``interesting
% suggestions,'' so you \textit{must not} ignore them.  Other issues are
% important too, so please pay attention to them as well.

%%%%%%%%%%%%%%%%%%%%%%%%%%%%%%%%%%%%%%%%%%%%%%%%%%%%%%%%%%%%%%%%%%%%%%%%%%%%%%

\section{L4 Syntax}

The lexical specification of L4 is changed by adding '\verb'['',
'\verb']'', '\verb'.'', and '\verb'->'' as lexical tokens; see
Figure~\ref{fig:tokens}.  Whitespace, token delimiting, and comments
are unchanged from languages L1, L2, and L3.

\begin{figure}
\begin{small}
\begin{tabular}{lcl}
\term{ident}      &::=& \verb"[A-Za-z_][A-Za-z0-9_]*"\\
\term{num}        &::=& \tok{decnum} $|$ \tok{hexnum}\\
\\
\tok{decnum}    &::=& \verb"0 | [1-9][0-9]*"\\
\tok{hexnum}    &::=& \verb"0[xX][0-9a-fA-F]+"\\
\\
\tok{special characters}
&::=& \verb"!  ~  -  +  *  /  %  <<  >>"   \\
&   & \verb"<  >  >= <= ==  !=  &  ^  |  &&  ||" \\
&   & \verb"=  +=  -=  *=  /=  %=  <<=  >>=  &=  |=  ^=" \\
&   & \verb"->  .  --  ++  (  |  )  [  ] , ;  ?  :  "   \\
\\
\tok{reserved keywords}
&::=& \verb"struct  typedef  if  else  while  for  continue  break" \\
&   & \verb"return  assert  true  false  NULL  alloc  alloc_array"  \\
&   & \verb"int  bool  void  char  string"               \\
\\
\end{tabular}
\end{small}

\medskip
Terminals referenced in the grammar are in \term{bold}.
Other classifiers not referenced within the grammar are
in \tok{angle brackets and in italics}. \term{ident},
\tok{decnum}, and \tok{hexnum} are described using
regular expressions.
\caption{Lexical Tokens}
\label{fig:tokens}
\end{figure}

The syntax of L4 is defined by the (no longer context-free!) grammar
in Figure \ref{fig:grammar-l4}.  Ambiguities in this grammar are
resolved according to the operator precedence table in
Figure~\ref{fig:precedence} and the rule that an \texttt{else}
provides the alternative for the most recent eligible \texttt{if}.

Again, note that this grammer is \emph{not} context free.  Just trying
to parse this language with a parser generator in a naive way will
\emph{not} result in correct behavior.  Some hints on how to parse
this language are provided in the next section.

\begin{figure}
\renewcommand{\arraystretch}{1.2}
\begin{small}
\begin{tabular}{lcl}
\nonterm{program} & ::= & $\epsilon$ \OR \nonterm{gdecl} \nonterm{program} \\

\nonterm{gdecl} & ::= & \nonterm{fdecl} \OR \nonterm{fdef} \OR \nonterm{typedef} \OR \nonterm{sdecl} \OR \nonterm{sdef} \\

\nonterm{fdecl} & ::= & \nonterm{type} \term{ident} \nonterm{param-list} \term{;} \\

\nonterm{fdef} & ::= & \nonterm{type} \term{ident} \nonterm{param-list} \nonterm{block} \\

\nonterm{param} & ::= & \nonterm{type} \term{ident} \\

\nonterm{param-list-follow} & ::= & $\epsilon$ \OR \term{,} \nonterm{param} \nonterm{param-list-follow} \\

\nonterm{param-list} & ::= & \term{( )} \OR \term{(} \nonterm{param} \nonterm{param-list-follow} \term{)} \\

\nonterm{typedef} & ::= & \term{typedef} \nonterm{type} \term{ident} \term{;} \\

\nonterm{sdecl} & ::= & \term{struct} \term{ident} \term{;} \\

\nonterm{sdef} & ::= & \term{struct} \term{ident} \verb"{" \nonterm{field-list} \verb"}" \term{;} \\

\nonterm{field} & ::= & \nonterm{type} \term{ident} \term{;} \\

\nonterm{field-list} & ::= & $\epsilon$ \OR \nonterm{field} \nonterm{field-list} \\

\nonterm{type} & ::= & \term{int} \OR \term{bool} \OR \term{ident} \OR \term{void}
\OR \nonterm{type} \term{*} \OR \nonterm{type}\term{[]} \OR \term{struct} \term{ident} \\

\nonterm{block} & ::= & {\bf \verb"{"} \nonterm{stmts} {\bf \verb"}"} \\

\nonterm{decl} & ::= & \nonterm{type} \term{ident} \OR \nonterm{type} \term{ident} \term{=} \nonterm{exp} \\

\nonterm{stmts} & ::= & $\epsilon$ \OR \nonterm {stmt} \nonterm {stmts} \\

\nonterm{stmt} & ::= & \nonterm{simp} \term{;} \OR \nonterm{control} \OR \nonterm{block} \\

\nonterm{simp} & ::= & \nonterm{lvalue} \nonterm{asop} \nonterm{exp} \OR \nonterm{lvalue} \nonterm{postop} \OR \nonterm{decl} \OR \nonterm{exp} \\

\nonterm{simpopt} & ::= & $\epsilon$ \OR \nonterm{simp} \\

\nonterm{lvalue} & ::= & \term{ident} \OR \nonterm{lvalue} \term{.} \nonterm{ident} \OR \nonterm{lvalue} \verb"->" \nonterm{ident} \\
& \OR & \term{*} \nonterm{lvalue} \OR \nonterm{lvalue} \term{[} \nonterm{exp} \term{]} \OR \term{(} \nonterm{lvalue} \term{)} \\

\nonterm{elseopt} & ::= & $\epsilon$ \OR \term{else} \nonterm{stmt} \\

\nonterm{control} & ::= & \term{if (} \nonterm{exp} \term{)} \nonterm{stmt} \nonterm{elseopt} \OR \term{while (} \nonterm{exp} \term{)} \nonterm{stmt} \\
& \OR & \term{for (} \nonterm{simpopt} \term{;} \nonterm{exp} \term{;} \nonterm{simpopt} \term{)} \nonterm{stmt} \\
& \OR & \term{return} \nonterm{exp} \term{;} \OR \term{return} \term{;} \\
& \OR & \term{assert (} \nonterm{exp} \term{)} \term{;} \\

\nonterm{arg-list-follow} & ::= & $\epsilon$ \OR \term{,} \nonterm{exp} \nonterm{arg-list-follow} \\

\nonterm{arg-list} & ::= & \term{( )} \OR \term{(} \nonterm{exp} \nonterm{arg-list-follow} \term{)} \\
\nonterm{exp} & ::= &
        \term{(} \nonterm{exp} \term{)}

  \OR   \term{num} \OR \term{true} \OR \term{false} \OR \term{ident} \OR \term{NULL} \OR \nonterm{unop} \nonterm{exp} \\
& \OR & \nonterm{exp} \nonterm{binop} \nonterm{exp} \OR \nonterm{exp} \term{?} \nonterm{exp} \term{:} \nonterm{exp} \OR \term{ident} \nonterm{arg-list} \\
& \OR & \nonterm{exp} \term{.} \term{ident} \OR \nonterm{exp} \verb"->" \term{ident} \OR \term{alloc (} \nonterm{type} \term{)} \OR \term{*} \nonterm{exp} \\
& \OR & \term{alloc\_array (} \nonterm{type} \term{,} \nonterm{exp} \term{)} \OR \nonterm{exp} \term{[} \nonterm{exp} \term{]} \\

\nonterm{asop} & ::= & \verb"=" \OR \verb"+=" \OR \verb"-=" \OR \verb"*=" \OR \verb"/=" \OR \verb"%="
\OR \verb"&=" \OR \verb"^=" \OR \verb"|=" \OR \verb"<<=" \OR \verb">>=" \\

\nonterm{binop} & ::= & \verb"+" \OR \verb"-" \OR \verb"*" \OR \verb"/" \OR \verb"%" \OR \verb"<"
\OR \verb"<=" \OR \verb">" \OR \verb">=" \OR \verb"==" \OR \verb"!=" \\
& \OR & \verb"&&" \OR \verb"||" \OR \verb"&" \OR \verb"^" \OR \verb"|" \OR \verb"<<" \OR \verb">>" \\

\nonterm{unop} & ::= & \verb"!" \OR \verb"~" \OR \verb"-" \\

\nonterm{postop} & ::= & \verb"++" \OR \verb"--"
\end{tabular}
\end{small}

\medskip
The precedence of unary and binary operators is given in
Figure~\ref{fig:precedence}.  Non-terminals are in \nonterm{angle
brackets}. Terminals are in \term{bold}.  The absence of tokens is
denoted by $\epsilon$.
\caption{Grammar of L4}
\label{fig:grammar-l4}
\end{figure}

\begin{figure}
\renewcommand{\arraystretch}{1.4}
\begin{tabular}{llll}
\hline
Operator & Associates & Meaning \\
\hline
\verb"() [] -> ."       & n/a   & explicit parentheses, array subscript, \\
                        &       & field dereference, field select \\
\verb"! ~ - * ++ --"      & right & logical not, bitwise not, unary minus, \\
					& & pointer dereference, increment, decrement \\
\verb"* / %"            & left  & integer times, divide, modulo \\
\verb"+ -"              & left  & integer plus, minus \\
\verb"<< >>"            & left  & (arithmetic) shift left, right\\
\verb"< <= > >="        & left  & integer comparison \\
\verb"== !="            & left  & overloaded equality, disequality \\
\verb"&"                & left  & bitwise and \\
\verb"^"                & left  & bitwise exclusive or \\
\verb"|"                & left  & bitwise or \\
\verb"&&"               & left  & logical and \\
\verb"||"               & left  & logical or \\
\verb"? :"              & right & conditional expression \\
\verb"= += -= *= /= %=" \\
\hspace{3em}\verb"&= ^= |= <<= >>=" & right & assignment operators \\
\hline
\end{tabular}
\caption{Precedence of operators, from highest to lowest}
\label{fig:precedence}
\end{figure}

%%%%%%%%%%%%%%%%%%%%%%%%%%%%%%%%%%%%%%%%%%%%%%%%%%%%%%%%%%%%%%%%%%%%%%%%%%%%%%

\section{L4 Elaboration}
\label{sec:elab}

We will not provide explicit elaboration rules for everything in
L4. Please note you will need to preserve at least some size
information during elaboration to generate correct code.

% \vspace*{10em}
\subsection*{Detail:  L4 syntax is not context-free}

As noted above, the grammar presented for L4 is no longer context free.
Consider, for example, the statement
\begin{verbatim}
  foo * bar;
\end{verbatim}
If \verb"foo" is a type name, then this is a declaration of a
\verb"foo" pointer named \verb"bar". If, however, \verb"foo" is
\emph{not} a type name, then this is a multiplication expression used
as a statement.

For those of you using parser combinator libraries, you will be able
to backtrack from a parse decision based on whether an identifier is a
type name, so this case should not be a problem.

However, those of you using parser generators will have a harder
time---the decision whether to shift or reduce could might be made
well ahead of when an identifier is determined to be a typename or
not. Solving this ambiguity is a bit tricky; below, we describe two
approaches.

One way to handle this is to perform an ambiguous parse: use one rule
to parse both the declaration form and the expression form (hint: an
existing rule may work!). Then undo an incorrect decision during
elaboration. This approach will almost certainly involve some other
adjustment to various pieces of your grammar and lexer. However, if
you are already performing elaboration in a seperate phase, we believe
this should be quite manageable.

Another option is to prevent incorrect decisions from being made. New
type identifiers are introduced at the top level (as a
\nonterm{gdecl}); you may wish to parse one global declaration at a
time, with suitable changes to the start-of-parse and end-of-parse
symbols). With this approach, the lexer can produce different tokens
for type and non-type identifiers.

This solves the parsing problem, but raises another. The lexer performs
a lookahead in order to find the longest match. This affects the lexing
of a token which is used immediately after it is introduced--consider:
\begin{verbatim}
  typedef int foo;
  foo func();
\end{verbatim}
In this case, if the parser parses \verb"typedef int foo;" the lexer
may already have lexed the \verb"foo" at the beginning of the next
line, so be careful! This approach may involve becoming more familiar
with the internals of the parser than you desire.

\section{L4 Static Semantics}

\subsection*{Top level declarations}

New in the grammar are struct declarations and definitions.  See
\href{http://www.cs.cmu.edu/~fp/courses/15411-f13/lectures/15-structs.pdf}{Lecture
  15} for some static semantics rules.  Not explicitly stated
there are the following:
\begin {itemize}
\item Struct \emph{definitions} obey scoping rules of the other global
  declarations, that is, they are available only after their point of
  definition.  However, structs may be \emph{declared} implicitly; see
  Section 2 of the notes.
\item Like type definitions, struct declarations and definitions can
  appear in external files.
\end {itemize}

\subsection*{Typechecking}

The type checking and static semantics rules for L4 are covered in the
notes for
\href{http://www.cs.cmu.edu/~fp/courses/15411-f13/lectures/14-mutable.pdf}{Lecture
  14} on \emph{Mutable Store} (for pointers and arrays) and
\href{http://www.cs.cmu.edu/~fp/courses/15411-f13/lectures/15-structs.pdf}{Lecture
  15} on \emph{Structs}. In addition:

\begin {itemize}
\item Postfix operators can be applied to destinations. However, there
  is an additional restriction that statements of the form
  ${*}d{\mbox{\tt ++}}$ and ${*}d{\mbox{\tt --}}$ are disallowed even
  though they would fit the grammar.  This is to avoid confusion with
  the different semantics such a statement would have in C.
\end {itemize}

\section{L4 Dynamic Semantics}

The dynamic semantics of L4 is covered in the notes for
\href{http://www.cs.cmu.edu/~fp/courses/15411-f13/lectures/14-mutable.pdf}{Lecture
  14} on \emph{Mutable Store} (for pointers and arrays) and
\href{http://www.cs.cmu.edu/~fp/courses/15411-f13/lectures/15-structs.pdf}{Lecture
  15} on \emph{Structs}.  Increment and decrement statements
\verb'd++;' and \verb'd--;' can be elaborated to \verb'd += 1;' and
\verb'd -= 1;', respectively.

For this lab, you should follow the data representations described in
the lecture notes.  Specifically, booleans should be represent as
4-byte values that are either 0 or 1.  This should guarantee
interoperability with the standard library.

For this lab, you do \emph{not} need to lay out structs in a way that
is compatible with C, but you are encouraged to do so.  You should
respect the machine's alignment requirements so that integers and
booleans are aligned at least 0 modulo 4 and addresses at least 0
modulo 8.  At this point we do not know if we will test this
provision.

%%%%%%%%%%%%%%%%%%%%%%%%%%%%%%%%%%%%%%%%%%%%%%%%%%%%%%%%%%%%%%%%%%%%%%%%%%%%%

\section{Project Requirements}

For this project, you are required to hand in test cases and a
complete working compiler for L4 that produces correct target
programs written in Intel x86-64 assembly language.

We also require that you document your code. Documentation includes
both inline documentation and a README document which explains the
design decisions underlying the implementation along with the general
layout of the sources. If you use publicly available libraries, you
are required to indicate their use and source in the README file. If
you are unsure whether it is appropriate to use external code, please
discuss it with course staff.

When we grade your work, we will use the \verb"gcc" compiler to
assemble and link the code you generate into executables using the
provided runtime environment on the lab machines.

Your compiler and test programs must be formatted and handed in via
Autolab as specified below. For this project, you must also write and
hand in at least 20 test programs, at least two of which must fail to
compile, at least two of which must generate a runtime error, and at
least two of which must execute correctly and return a value.

\subsection*{Test Files}

Test programs should have extension \verb".l4" and start with one of the
following lines

\begin{tabular}{l@{\hspace{5em}}l}
\verb"//test return "$i$ & program must execute correctly and return $i$ \\
\verb"//test exception "$n$ & program must compile but raise runtime exception $n$ \\
\verb"//test exception " & program must compile but raise \emph{some} runtime exception \\
\verb"//test error" & program must fail to compile due to an L4 source error
\end{tabular}

\noindent followed by the program text.  In L4, the exceptions defined
are \verb"SIGABRT" (6) (assertion failure), \verb"SIGFPE" (8)
(arithmetic exception), \verb"SIGSEGV" (11) (memory-related exception)
and \verb"SIGALARM" (14) (timeout).

Since the language now supports function calls, the runtime
environment contains external functions providing output capabilities
(see the runtime section for details).  The testing framework may
decide to take advantage of this.  For a file \texttt{\$test.l4}:
\begin{itemize}
% \item If a file \texttt{\$test.l3.in} exists, its contents will be available to
% the program as input during testing.
\item If a file \texttt{\$test.l4.out} exists \emph{and the program returns}, the
output of the program must be identical to the contents of the file for the
test to pass.
\end{itemize}
Please use this capability sparingly, as the behavior of some library
functions may be implementation dependent.

All test files should be collected into a directory \verb"tests/"
(containing no other files) and submitted via the Autolab server.

L4 is a sophisticated and reasonably expressive language. You should
be able to write some very interesting tests, perhaps adapted from the
programs and libraries you wrote in the 15-122 course on
\emph{Principles of Imperative Computation} that uses C0.

Please do \emph{not} submit test cases which differ in only small ways
in terms of the behavior exercised. We would like some fraction of
your test programs to compute ``interesting'' functions on specific
values; please briefly describe such examples in a comment in the
file. Disallowed are programs which compute Fibonacci numbers,
factorials, greatest common divisors, and minor variants
thereof. Please use your imagination!

\subsection*{Compiler Files}

The files comprising the compiler itself should be collected in a
directory \verb"compiler/" which should contain a \verb"Makefile".
\textbf{Important:} You should also update the \verb"README" file and
insert a description of your code and algorithms used at the beginning
of this file.  Even though your code will not be read for grading
purposes, we may still read it to provide you feedback.  The
\verb"README" will be crucial information for this purpose.

Issuing the shell command
\begin{verbatim}
  % make l4c
\end{verbatim}
should generate the appropriate files so that
\begin{verbatim}
  % bin/l4c <args>
\end{verbatim}
will run your L4 compiler.  The command
\begin{verbatim}
  % make clean
\end{verbatim}
should remove all binaries, heaps, and other generated files.

\subsection*{Runtime Environment}

Your compiler should accept a command line argument \texttt{-l} which
must be given the name of a file as an argument. For instance, we will
be calling your compiler using the following command: \texttt{bin/l4c
  -l 15411.h0 \$test.l4}. Here, \texttt{15411.h0} is the header file
mentioned in the elaboration and static semantics sections.

The runtime for this lab contains functions to perform output. The
output functions write to \texttt{stderr}, while the result of the
program is printed to \texttt{stdout}. \texttt{15411.h0} contains a
listing and a small amount of documentation.

The GNU compiler and linker will be used to link your assembly to the
implementations of the external functions, so you need not worry much
about the details of calling to external functions. You should ensure
that the code you generate adheres to the C ABI for Linux on
x86-64.  In order for the linking to work, you must adhere to the
following conventions
\begin{itemize}
\item External functions must be called as named.
\item Non-external functions with name $\mathit{name}$ must be called
  \verb'_c0_'$\mathit{name}$.  This ensures that non-external function
  names do not accidentally conflict with names from standard library
  which could cause assembly or linking to fail.
\item Non-external functions must be exported from (declared to be
  \emph{global} in) the assembly file you generate, so that our test
  harness can call them and verify your adherence to the ABI.
\end{itemize}

The runtime environment defines a function \verb"main()" which calls a
function \verb"_c0_main()" your assembly code should provide and export.  Your
compiler will be tested in the standard Linux environment on the lab machines;
the produced assembly must conform to this environment.

\subsection*{Using the Subversion Repository}

Handin and handout of material is via the course subversion
repository.

The handout files for this course can be checked out from our
subversion repository via
\begin{verbatim}
  % svn checkout https://svn.concert.cs.cmu.edu/15411-f13/groups/<team>
\end{verbatim}
where \verb"<team>" is the name of your team.  You will find materials
for this lab in the \verb"lab4" subdirectory.  Or, if you have checked
out \verb"15411-f13/groups/<team>" directory before, you can issue the
command \verb"svn update" in that directory.

After adding and committing your handin directory to the
repository with \verb"svn add" and \verb"svn commit"
you can hand in your tests or compiler through Autolab.
\begin{verbatim}
  https://autolab.cs.cmu.edu/15411-f13
\end{verbatim}
Once logged into Autolab, navigate to the appropriate assignment and
select
\begin{verbatim}
  Checkout your work for credit
\end{verbatim}
from the menu.  It will perform one of
\begin{verbatim}
  % svn checkout https://svn.concert.cs.cmu.edu/15411-f13/groups/<team>/lab4/tests
  % svn checkout https://svn.concert.cs.cmu.edu/15411-f13/groups/<team>/lab4/compiler
\end{verbatim}
to obtain the files directories to autograde, depending on whether
you are handing in your test files or your compiler.

If you are submitting multiple versions, please remember to commit
your changes to the repository before asking the Autolab server to
grade them!  And please do not include any compiled files or binaries
in the repository!

\subsection*{What to Turn In}

Hand in on the Autolab server:

\begin{itemize}

\item At least 20 test cases, at least two of which generate an error,
  at least two of which raise a runtime exception, and at least two of
  which return a value.  The directory \verb'tests/' should only
  contain your test files and be submitted via subversion as described
  above.  The server will test your test files and notify you if there
  is a discrepancy between your answer and the outcome of the
  reference implementation.  You may hand in as many times as you like
  before the deadline without penalty.  If you feel the reference
  implementation is in error, please notify the instructors.  The
  compiled binary for each test case should run in 2 seconds with the
  reference compiler on the lab machines; we will use a 5 second limit
  for testing compilers.

  Test cases are due \textbf{11:59pm on Thursday, October 24, 2013}.

\item The complete compiler.  The directory \verb'compiler/' should
  contain only the sources for your compiler and be submitted via
  subversion.  The Autolab server will build your compiler, run it on
  all extant test files, link the resulting assembly files against our
  runtime system (if compilation is successful), execute the binaries
  (each with a 5 second time limit), and finally compare the actual
  with the expected results.  You may hand in as many times as you
  like before the deadline without penalty.

  Compilers are due \textbf{11:59pm on Thursday, October 31, 2013}.

\end{itemize}

%%%%%%%%%%%%%%%%%%%%%%%%%%%%%%%%%%%%%%%%%%%%%%%%%%%%%%%%%%%%%%%%%%%%%%%%%%%%%

\section{Notes and Hints}

The rules for struct declarations and definitions are carefully
engineered so that it should be possible to compute the size and field
offsets of each struct without referring to anything found later in
the file. You probably want to store the sizes and field offsets in
global tables.

Data now can have different sizes, and you need to track this
information throughout the various phases of compilation.  We suggest
you read Section 4 of the Bryant/O'Hallaron notes on
\href{http://www.cs.cmu.edu/~fp/courses/15411-f13/misc/asm64-handout.pdf}{x86-64
  Machine-Level Programming} available from the course Resources page,
especially the paragraph on move instructions and the effects of 32
bit operators in the upper 32 of the 64 bit registers.  Some notes can
also be found in Section 6 of the notes for Lecture 15.

\end{document}
