\subsection{Tokens}

%\begin{figure}
\begin{small}
\begin{tabular}{lcl}
\term{ident}      &::=& \verb"[A-Za-z_][A-Za-z0-9_]*[']*"\\
\term{num}        &::=& \tok{decnum} \OR \tok{hexnum}\\
\\
\tok{decnum}    &::=& \verb"0 | [1-9][0-9]*"\\
\tok{hexnum}    &::=& \verb"0x[1-9a-f][0-9a-f]*"\\
\\
\tok{integer binary operators}
&::=& \verb"-  +  *  /  %  **"   \\
\\
\tok{comparison operators}
&::=& \verb"<  >  >= <= ==  !=" \\
\\
\tok{other operators}
&::=& \verb"\  @  &" \\
\\
\tok{other tokens}
&::=& \verb"->  <-  ...  ^  (  )  [  ] ,  :  =>  =  ||  .|  .:  _"   \\
\\
\tok{unary operators}
&::=& \verb"!  ~  #"   \\
\\
\tok{constructors}
&::=& \verb".<  >.  .{  }."\\
\\
\tok{reserved keywords}
&::=& \verb"if  then  else let  val  in  end  case  of  fun  \in" \\
&   & \verb"assert  true  false  domain  range"  \\
&   & \verb"int  bool  set  table  seq  graph" \\
&   & \verb"\min  \max  \argmax  \argmin  \sum  \union  \intersect"
\\
\end{tabular}
\end{small}

\medskip
Terminals referenced in the grammar are in \term{bold}.
Other classifiers not referenced within the grammar are
in \tok{angle brackets and in italics}. \term{ident}, \tok{hexnum} and \tok{decnum} are described using
regular expressions.\\
%\caption{Lexical Tokens}
%\label{fig:tokens}
%\end{figure}
